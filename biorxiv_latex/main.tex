\documentclass[12pt]{biorxiv}

\usepackage[english]{babel}
\usepackage[utf8x]{inputenc}
\usepackage{amsmath}
\usepackage{amssymb}
\usepackage{graphicx}
\usepackage{subcaption}
\usepackage{booktabs} \newcommand{\ra}[1]{\renewcommand{\arraystretch}{#1}}
\usepackage{multirow}
\usepackage[figuresfirst,nolists,nomarkers]{endfloat}
\usepackage[colorinlistoftodos,textsize=tiny]{todonotes}
\usepackage{lineno}
\usepackage{setspace}
\usepackage[scaled]{DejaVuSansMono}
\usepackage[T1]{fontenc}  
\usepackage{xcite}
\usepackage{xr}
\usepackage{color}
\usepackage{breakcites}

%\linenumbers
%\pagenumbering{gobble}

%\title{LATEX Template - bioRxiv}
\title{The real title of this manuscript}
\author[1]{Minion 1}
\author[1]{Minion 2}
\author[1,*]{Gary A. Churchill}
\affil[1]{The Jackson Laboratory, 600 Main Street, Bar Harbor, Maine, 04609}

%
% START HELPER CODE
%
\makeatletter
\newcommand*{\addFileDependency}[1]{% argument=file name and extension
  \typeout{(#1)}% latexmk will find this if $recorder=0 (however, in that case, it will ignore #1 if it is a .aux or .pdf file etc and it exists! if it doesn't exist, it will appear in the list of dependents regardless)
  \@addtofilelist{#1}% if you want it to appear in \listfiles, not really necessary and latexmk doesn't use this
  \IfFileExists{#1}{}{\typeout{No file #1.}}% latexmk will find this message if #1 doesn't exist (yet)
}
\makeatother

\newcommand*{\myexternaldocument}[1]{%
    \externaldocument{#1}%
    \addFileDependency{#1.tex}%
    \addFileDependency{#1.aux}%
}

% \makeatletter
% \newcommand*\bigcdot{\mathpalette\bigcdot@{.5}}
% \newcommand*\bigcdot@[2]{\mathbin{\vcenter{\hbox{\scalebox{#2}{$\m@th#1\bullet$}}}}}
% \makeatother

% \renewcommand{\thesubfigure}{\Alph{subfigure}}  % subfigure bullets

%
% END HELPER CODE
%

% put all the external documents here!
\myexternaldocument{supplementary}


\begin{document}

\maketitle
\begin{spacing}{2}

\begin{abstract}
Put your abstract here 
\end{abstract}


% Include a list of up to six keywords after the abstract
\keywords{Collaborative Cross, proteomics, genetics}

% Include email contact information for corresponding author
{ \noindent \footnotesize \textbf{*}\linkable{Gary.Churchill@jax.org} }



\vspace{0.5in}

Introduction here~\cite{Keele2020}.


\section*{Results}

Here are our novel findings.

\subsection*{The first finding}

We did this first.


\subsection*{The second finding}

We did this next.


\subsection*{The third finding}

We did this next.



\section*{Discussion}

In summary, we find substantial evidence for $\cdots$



\section*{Methods}

\subsection*{Data}

Proteins


\subsection*{Linear mixed models}

We used blah blah.


\subsection*{Software}
We have implemented our model 




%
%
% Figures
%
%

% Figures here


%
%
% Tables
%
%

% Tables here



\end{spacing}


%
%
% References
%
%
\newpage
\bibliographystyle{genres}
\bibliography{references}



\end{document}

